\documentclass{standalone}
% \documentclass{article}
% \documentclass{beamer}
% \usetheme{Pittsburgh}
% \usetheme{Rochester}
% \usecolortheme{spruce}

% \usepackage{mathtools}
% \usepackage{amsmath}
% \usepackage{amssymb}
% \usepackage{braket}
\usepackage{tikz}
% \usepackage[tikz]{bclogo} 
\usepackage[tikz]{bclogo} 
\usepackage{braket}
\usetikzlibrary{quantikz}
\begin{document}
% \begin{quantikz}
% \lstick{\ket{0}} & \phase{\alpha} & \gate{H}
% & \phase{\beta} & \gate{H} & \phase{\gamma}
% & \rstick{Arbitrary\\pure state}\qw
% \end{quantikz}
% \begin{quantikz}
% \ket0 & \gate{g_1}
% & \gate{\cdots}
% & \gate{g_m}
% & \gate{(g_1\circ \cdots \circ g_m)^{-1}}
% & \gate{\bra0}\qw
% \end{quantikz}
%\begin{quantikz}[slice all]
% \lstick{\(\ket{x_0}\)} & \gate{H} & \ctrl{1} & \ctrl{2} & \qw      & \qw        & \qw     & \qw      & \rstick{\(\ket0 + e^{2\pi i x/2}\ket1\)} \\
% \lstick{\(\ket{x_1}\)} & \qw   & \control{}  & \qw      & \gate{H} & \ctrl{1}   & \qw  & \qw     & \rstick{\(\ket0 +e^{2\pi i x/4} \ket1\)} \\
% \lstick{\(\ket{x_2}\)} & \qw   & \qw      &  \control{} &  \qw     & \control{} & \gate{H} & \qw & \rstick{\(\ket0 +e^{2\pi i x/8} \ket1\)} 
%\lstick{\(\ket\psi\)} & \ctrl1  & \ctrl2 &\gate[wires=3]{E} & \ctrl1 &\ctrl2 &\targ{}&\qw\rstick{\(\ket\psi\)}\\
%\lstick{\(\ket0\)}     & \targ{} & \qw    &  &\targ{}&\qw &\ctrl{-1} &\qw\\
%\lstick{\(\ket0\)}     & \qw     & \targ{} & & \qw & \targ{} & \ctrl{-2}&\qw
%\end{quantikz}

\begin{quantikz}[]
    % \lstick{\(\ket{x_0}\)} & \gate{H} & \ctrl{1} & \ctrl{2} & \qw      & \qw        & \qw     & \qw      & \rstick{\(\ket0 + e^{2\pi i x/2}\ket1\)} \\
    % \lstick{\(\ket{x_1}\)} & \qw   & \control{}  & \qw      & \gate{H} & \ctrl{1}   & \qw  & \qw     & \rstick{\(\ket0 +e^{2\pi i x/4} \ket1\)} \\
    % \lstick{\(\ket{x_2}\)} & \qw   & \qw      &  \control{} &  \qw     & \control{} & \gate{H} & \qw & \rstick{\(\ket0 +e^{2\pi i x/8} \ket1\)} 
    \lstick{\(\ket\psi\)} & \ctrl1  & \gate{H} &  \qw  \\
    \lstick{\(\ket0\)}    & \targ{} & \qw      & \gate{T}              
\end{quantikz}

% \begin{quantikz}
% \lstick{$\ket{0}$} & \gate{H} & \ctrl{1} & \gate{U}   & \ctrl{1}   & \swap{2} & \ctrl{1} & \qw \\
% \lstick{$\ket{0}$} & \gate{H} & \targ{}  & \octrl{-1} & \control{} & \qw      & \octrl{1} & \qw \\
%                    &          &          &            &            &\targX{}  & \gate{U} & \qw
% \end{quantikz}
% \begin{frame}
%     \frametitle{Basics: the method}
%     \begin{itemize}
%         \item Define a ``mapping'' between group elements \(g\)
%         and gates \(\boxed{g}\).
%     \item Randomly sample \(m\) group members \(g_1,\dots, g_m\).
%     \item Randomised Benchmarking run.
%     \begin{quantikz}
%         \ket0
%         % \lstick{\(\ket0\)}% & \gate{g_1}
%         % \lstick{\(\ket0\)}% & \gate{g_1}
%         % & \gate{\cdots}
%         % & \gate{g_m}
%         % & \gate{(g_1\circ \cdots \circ g_m)^{-1}}
%         % & \gate{\bra0}\qw
%         % & \rstick{Arbitrary\\pure state}\qw
%         \end{quantikz}
%     \item For a mathematical description of the group element \(g_i\), 
%     I will use \(G(g_i)\). \(G\) is a \textbf{map} between a group and a list of gates.
%     \item Thus, the \emph{analytical value} for the circuit is 
%     \begin{equation}
%         \bra0 (G((g_1 \cdots g_m)^{-1})) \circ G(g_m)\circ\cdots\circ G(g_2)\circ G(g_1)(\ket0)
%     \end{equation}
%     \end{itemize}

    

% \end{frame}

\end{document}