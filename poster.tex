\documentclass[12pt]{article}
\usepackage[a3paper,landscape]{geometry}

\usepackage[poster]{tcolorbox}
\usepackage{pagecolor}
\usepackage{background}
\usepackage{amsmath}
\usepackage{mdframed}
\pagestyle{empty}
\pagecolor{red}

\begin{document}












    



    









































    




    

    
















    





    

    
    
    
    















\begin{tcbposter}[
coverage = {spread, interior style={top color=cyan,bottom color=cyan!50!red} },
poster = {columns=3,rows=2},
boxes = {fonttitle = \bfseries\Large\scshape, sharp corners = downhill, arc=3mm, boxrule = 1mm},
fontsize = 16pt
]
\posterbox[adjusted title = Randomised Benchmarking of universal qutrit gates, colframe = teal!50!black, colback=teal!50,valign=center]{
    column = 1,
    name=project, 
    between=top and bottom
}{
{
% \vspace*{20mm}
Explicit Randomised Benchmarking qutrit schemes are limited to Clifford gates.
We introduce a \textbf{\large scheme to characterise a qutrit T gate}.

\begin{center}
\includegraphics[width=0.9\textwidth]{auxiliary figures/matrices.pdf}
\end{center}

Our scheme is
a \textbf{\large feasible} extension to qutrits of the Dihedral Benchmarking scheme 10/brjj.
Our scheme is the synthesis of the \textbf{\large Fourier method} 10/jrwr applied to non-Clifford gates.


Our scheme is important for experimental groups with qutrit implementations 10/gj8dt4, 
theorists working on Randomised Benchmarking methods, and in general theorists 
interested in the application of Representation Theory in Physics.
}





}
\posterbox{name=title,column=1,above=bottom}{
    \underline{David Amaro-Alcala}, Barry C. Sanders, Hubert de Guise.
    DAA is supported by the Government of Alberta and NSERC.
}




\posterbox[adjusted title = Background, valign = center]{
    name=result, 
    column = 2,row =1,span = 2
}{
\begin{minipage}{0.49\linewidth}





Randomized Benchmarking estimates quantum gate quality by comparing the behaviour
of ideal and noisy gates, via the average gate fidelity $F$ 10/tfz.
\includegraphics[width=\textwidth]{auxiliary figures/crazy-scientist.pdf}
\end{minipage}
\hspace{0.02\textwidth}
\begin{minipage}{0.45\linewidth}





RB is used to characterize Clifford gates, T gates require an extension 
of the RB scheme for their characterization.
\begin{center}
\includegraphics[width=0.45\textwidth]{auxiliary figures/universal-circuit.pdf}
\end{center}



A \textbf{qutrit} is a three-level quantum system that offers advantages over qubits and
is widely available in different quantum information implementations 10/ghptsj.
\end{minipage}
}



\posterbox[adjusted title = Results,valign = center]{
% \posterbox[adjusted title = Results]{
    name=result, 
    column = 2,row =2,span = 2
}{
\begin{minipage}{0.45\linewidth}




RB assumes gates correspond to a physical group; we introduce the HyperDihedral group to characterise a T gate.
The HyperDihedral group is generated by \(X,T\), and \(D\).
\begin{center}
\includegraphics[width=0.9\textwidth]{auxiliary figures/simulation.pdf}
\end{center}
\end{minipage}
\hspace{0.02\textwidth}
\begin{minipage}{0.45\linewidth}
The name HyperDihedral is given because it is a
generalisation of the Dihedral group.
We obtained the expression for the average gate fidelity 
for the HyperDihedral group;
it has two parameters, accessible by using two different initial states.
Our expression 
\begin{equation*}
    \textstyle F = \frac{3}{4}(1+2r_0 + 6 r_0) + \frac{1}{4}.
\end{equation*}
is valid for state imperfections and gate-dependent errors.
\end{minipage}



}
\end{tcbposter}
\end{document}